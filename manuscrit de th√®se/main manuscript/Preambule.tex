
%%%%%%%%%%%%%%%%%%%%%%%%%%%%%%%%%%%%%%%%%%%%%%%%%%%%%%%%%%%%%%%%%%%%%%
%\usepackage[square,sort&compress,sectionbib]{natbib}		% Doit être chargé avant babel
%\usepackage[square,sort,comma,numbers]{natbib}
%\usepackage{chapterbib}
%	\renewcommand{\bibsection}{\section{Références}}		% Met les références biblio dans un \section (au lieu de \section*)
%		

%\usepackage{lmodern}
%\usepackage{ae,aecompl}										% Utilisation des fontes vectorielles modernes
%\usepackage[upright]{fourier}
%
%
%
%%%%%%%%%%%%%%%%%%%%%%%%%%%%%%%%%%%%%%%%%
%%           Liste des packages         %
%%%%%%%%%%%%%%%%%%%%%%%%%%%%%%%%%%%%%%%%%

\usepackage[english, francais]{babel}  
\addto\captionsenglish{% Replace "english" with the language you use
  \renewcommand{\contentsname}%
    {Table of contents}%
}
\usepackage[utf8]{inputenc}
\usepackage[T1]{fontenc}
\usepackage{cite}

%\usepackage[babel=true]{csquotes}

\usepackage{color}

\usepackage[normalem]{ulem}
\usepackage{soul}
\usepackage{xcolor}
\usepackage[colorlinks=true,linkcolor=black,citecolor=black,urlcolor  = black,citecolor = black,pagebackref,anchorcolor = black]{hyperref}

\usepackage[english]{minitoc}
\mtcsettitle{minitoc}{Outline}
\usepackage{textcomp}
\usepackage{lmodern}
\usepackage{xspace}
\usepackage[final]{pdfpages}
\usepackage[toc,page]{appendix}
\usepackage{geometry}  
\usepackage[width=0.9\textwidth]{caption}
\usepackage{array}
\usepackage{url}
\usepackage{breakurl}
\usepackage{fancyhdr}
% permet de faire une table des matieres par chapitre
\usepackage{lipsum}
\usepackage{afterpage}


%%%%%%%%%%%%%%%%%%%%%%%%%%%%%%%%%%%%%%%%%
%%           new commands         %
%%%%%%%%%%%%%%%%%%%%%%%%%%%%%%%%%%%%%%%%%

%\newcommand{\com}[1]{{\color{red}\emph{#1}}}
\definecolor{lblue}{rgb}{0,0,0.2}

%\newcommand{\alp}{\texorpdfstring{\ensuremath{\upalpha}\xspace}{alpha }}
%\newcommand{\bet}{\texorpdfstring{\ensuremath{\upbeta}\xspace}{b\'{e}ta }}
%\newcommand{\alpbet}{\texorpdfstring{\ensuremath{\upalpha-\upbeta}\xspace}{alpha-b\'{e}ta}}
%\newcommand{\alpt}{\ensuremath{\alpha_2}\xspace}
%\newcommand{\strt}{\gls{strt}\xspace}



\newcommand{\g}[1]{\textbf{#1}}  % à mettre en valeur
\newcommand{\T}[1]{\underline{#1}}
\newcommand{\TT}[1]{\underline{\underline{#1}}}
\newcommand{\TTT}[1]{\underline{\underline{\underline{#1}}}}

% inserting a blank page
%\newcommand\blankpage{%
%    \null
%    \thispagestyle{empty}%
%    \addtocounter{page}{-1}%
%    \newpage}

%
%%%%%%%%%%%%%%%%%%%%%%%%%%%%%%%%%%%%%%%%%%%%%%%%%%%%%%%%%%%%%%%%%%%%%%



%%%%%%%%%%%%%%%%%%%%%%%%%%%%%%%%%%%%%%%%%
%%           Apparence         %
%%%%%%%%%%%%%%%%%%%%%%%%%%%%%%%%%%%%%%%%%          
\pagestyle{fancy} % or headings
\fancyhf{}
\renewcommand{\chaptermark}[1]{\markboth{\bsc{\chaptername~\thechapter{} :} #1}{}}
\renewcommand{\sectionmark}[1]{\markright{\thesection{} \ #1}}
\renewcommand{\headrulewidth}{0.4pt}
\lhead[]{\textsl{\rightmark}}
\rhead[\textsl{\leftmark}]{}
\fancyhead[R]{\thepage}
\fancyheadoffset{7mm}


% line spacing
\renewcommand{\baselinestretch}{1.2}

%\NoAutoSpaceBeforeFDP
%%% debut comment

%%%%%%%%%%%%%%%%%%%%%%%%%%%%%%%%%%%%%%%%%
%%%                                                   margin of document
%%%%%%%%%%%%%%%%%%%%%%%%%%%%%%%%%%%%%%%%%
%
 \geometry{
textwidth = 159.2mm,
textheight =690pt,
 }
\setlength{\voffset}{10mm}
\setlength{\hoffset}{-1mm}
\setlength{\headsep}{10mm}
\setlength{\footskip}{1cm}




%%%%%%%%%%%%%%%%%%%%%%%%%%%%%%%%%%%%%%%%%%%%%%%%%%%%%%%%%%%%%%%%%%%%%%
%
%%% Maths                         
\usepackage{amsmath}			% Permet de taper des formules mathématiques
\usepackage{amssymb}			% Permet d'utiliser des symboles mathématiques
\usepackage{amsfonts}			% Permet d'utiliser des polices mathématiques
\usepackage{nicefrac}
\usepackage{upgreek}			% For roman (i.e. upright) lowercase Greek characters
%
%%%%%%%%%%%%%%%%%%%%%%%%%%%%%%%%%%%%%%%%%%%%%%%%%%%%%%%%%%%%%%%%%%%%%%
%
%
%%% Tableaux
\usepackage{multirow}
\usepackage{booktabs}
\usepackage{colortbl}
\usepackage{tabularx}
\usepackage{multirow}
\usepackage{threeparttable}
\usepackage{etoolbox}
%\appto\TPTnoteSettings{\footnotesize}
%\addto\captionsfrench{\def\tablename{{\textsc{Tableau}}}}	% Renome 'table' en 'tableau'
%
%            
%            
%
%%%%%%%%%%%%%%%%%%%%%%%%%%%%%%%%%%%%%%%%%%%%%%%%%%%%%%%%%%%%%%%%%%%%%%
%%% Graphiques         
           
\usepackage{subfig}
\usepackage{graphicx}
\usepackage{epstopdf}

%\usepackage{subcaption}
%\usepackage{pdfpages}
%\usepackage{rotating}
%\usepackage{pgfplots}
%	\usepgfplotslibrary{groupplots}
%\usepackage{tikz}
%	\usetikzlibrary{backgrounds,automata}
%	\pgfplotsset{width=7cm,compat=1.3}
%	\tikzset{every picture/.style={execute at begin picture={
%   		\shorthandoff{:;!?};}
%	}}
%	\pgfplotsset{every linear axis/.append style={
%		/pgf/number format/.cd,
%		use comma,
%		1000 sep={\,},
%	}}
%\usepackage{eso-pic}
%\usepackage{import}
%\usepackage{cclicenses}
%
%%%%%%%%%%%%%%%%%%%%%%%%%%%%%%%%%%%%%%%%%%%%%%%%%%%%%%%%%%%%%%%%%%%%%%
%% Biblio                        
%%\makeatletter
%%\patchcmd{\BR@backref}{\newblock}{\newblock(page~}{}{}	% Pour les back-references, affiche 'page' au lieu de 'p.'
%%\patchcmd{\BR@backref}{\par}{)\par}{}{}
%%\makeatother
%	
%	
%%%%%%%%%%%%%%%%%%%%%%%%%%%%%%%%%%%%%%%%%%%%%%%%%%%%%%%%%%%%%%%%%%%%%%
%%% Navigation dans le document   
%\usepackage[pdftex,pdfborder={0 0 0},
%			colorlinks=true,
%			linkcolor=blue,
%			citecolor=red,
%			pagebackref=true,
%			]{hyperref} %Créera automatiquement les liens internes au PDF
%
%
%%%%%%%%%%%%%%%%%%%%%%%%%%%%%%%%%%%%%%%%%%%%%%%%%%%%%%%%%%%%%%%%%%%%%%

%%%%%%%%%%%%%   Rajouter en plus du template pour moi:  %%%%%%%%%%%%%
\usepackage{lipsum}
\usepackage{scrhack}
\usepackage{amssymb}
\usepackage{mathrsfs,amsthm}
\usepackage{calrsfs}
\usepackage{mathtools}
\usepackage[overload]{empheq}
\usepackage{cases} 
\usepackage{multicol}
\usepackage{makeidx}
\usepackage[nottoc]{tocbibind} % ajoute (entre autre) la bibliographie dans la table des matieres 
\usepackage{float}
\usepackage{braket}

%%%%%%%%%%%%%%%%%%%%%%%%%%%%%%%%%%%%%%%%%%%%%%%%%%%%%%%%%%%%%%%%%%%%%%


%%% Line counting option for rapporteurs - line number:

%%\usepackage[right]{lineno}
%%\linenumbers

