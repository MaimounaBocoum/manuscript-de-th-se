\chapter{Conclusion and perspectives}
\thispagestyle{empty}

\noindent \g{Conclusion:}\\

Several key experimental results have been obtained during this PhD. First, after a brief introduction of the Brunel mechanism and Coherent Wake Emission, we have presented a series of experiments related to high-order-harmonic generation from plasma mirrors. In particular, we saw how a variation of the consecutive emission times inside an attosecond pulse train has a direct influence on the spectral and spatial properties of the harmonic emission (spectral broadening, spatial modulations and divergence). Since most of those properties have already been described in the literature, we chose to present only experimental results bringing new insight into the understanding of the plasma dynamics. \\
\indent For instance, we confirmed the drop in HHG signal at very short gradient scale length ($<<\lambda /100$) and attribute this effect to Landau damping effects due to strong inhomogeneities. Still on HHG emission, we investigated the possible use of FROG-type retrieval algorithms for the temporal reconstruction of the attosecond train emitted in the presence of laser wavefront rotation. On the one hand, we have demonstrated that it was possible to perform a time-to-space mapping of the harmonic emission, and observed an incease in the harmonic femtosecond chirp with increasing plasma scale lengths. This was manifested by a tilt of the angularly-resolved harmonic line spectra. In addition, we successfully  implemented a FROG-like algorithm on the numerically generated attosecond traces. However, those traces were generated making the strong hypothesis that they can be written as the product of a space dependent function times a spectral dependent function. This property is a priori not true for the experimental traces, which, in addition to strong noise and signal clipping on the MCP,  made the actual reconstruction impossible with experimental data.\\
\indent Other experimental results presented in this manuscript concern the simultaneous detection of electrons and HHG as a function of the gradient scale length. It was already known that the HHG emission drops and that electron ejection increases with increasing plasma scale lengths. Here we show for the first time that their relative variation have opposite dependence with respect to the gradient. This transition is due to a modification of the electronic structure at the surface of the plasma mirror where space-charge effects play an increasing role. \\
\indent Finally, we discussed the spatial profile of the ejected electron beam. We observed the ponderomotive hole formed by the laser and how it disappears when L increases. We attribute this to a complete destruction of the laser reflected phase front. We have also discussed the gyromagnetic effect, which we believe prevents electrons from escaping the target at high intensity and for short gradient scale lengths. In strong focusing geometries and at high laser intensities, we have shown that the electron profile exhibits a new symmetry with respect to the plane of polarization of the incident laser.\\



\noindent \g{Future directions:}\\

All our experiments  have been conducted with a $\sim 30\,\mathrm{fs}$ laser pulse with sub-relativistic intensities. The laser is currently being upgraded to produce $5\,\mathrm{fs}$ pulses on target with the same energy. This will allow us to investigate electron acceleration and ROM-HHG emission from plasma mirrors at relativistic intensities. In addition, the non-linear interactions will be sensitive to the Carrier Envelope Phase (CEP) of the few-cycle pulse, which should allow us to separate attosecond pulses with higher photon energy content, and simultaneously generate electrons with energy of a few MeV. \\


\noindent We already mentioned in the introduction the existence of XFEL sources, which can provide short radiation pulses from UV to hard X-rays (few keV) of few $\sim$mJ energy. HHG from plasma mirrors are far from replacing these type of sources today, because the pulses are only on the $\sim$nJ level. However, it can be regarded as complementary and low-cost compared to XFELs. Here are some of the advantages we can stress in favor of ultrashort X-UV pulses generation from plasma mirrors:\\

\begin{itemize}
\item[$\bullet$] \g{Source tunability:} HHG on plasma mirrors can easily be tuned both spatially and temporally~\cite{Borot2012} because the  generated X-UV inherits the coherence properties of the IR pulse used to drive the interaction. In other words, instead of using XUV optics, which are difficult to find in general, one can for instance shape the laser spatial properties to "mark" the harmonics.  Note that this tunability is also true for the HHG in gases. One example of this is the generation of harmonics with orbital angular momentum~\cite{geneaux2015attosecond}.

\item[$\bullet$] \g{Separation of attosecond pulses:} Another demonstration of the XUV tunability is the possibility of separating attosecond pulses from one another using STCs on the IR pulse. So far, the attosecond lighthouse has only been demonstrated in the CWE regime~\cite{Wheeler2012}. In future experiments, we will try to isolate XUV pulses from ROM emission and perform XUV-XUV pump-probe experiments with attosecond resolution.

\item[$\bullet$] \g{Ultrafast XUV non-linear optics:} Two-photon ionization experiments in the X-UV region with high temporal resolution have been performed~\cite{miyamoto2004observation,guan2008dynamics,papadogiannis2003two}. However, they remain challenging because of low signal-to-noise ratio. If we manage to significantly increase the energy per pulse of the generated ultrashort XUV pulses, we will be able to study the two-photon ionization process on subfemtosecond time scales.

\item[$\bullet$] \g{High spatial resolution:} Strong focusing of X-UV light for highly resolved imaging is very challenging and relies mostly on the conception of aberration-free demagnification optics~\cite{poletto2013micro,valentin2003imaging}. One advantage of HHG from PM over HHG in gases is the possibility of generating X-UV light with focal spot sizes reaching the diffraction limit of the driving laser. The X-UV source has a very small focal spot size itself, and therefore can be tightly refocused using a simple 2f-2f imaging system. This will allow us to perform ultrafast time-resolved experiments with high spatial resolution. 
\end{itemize}

\vspace{0.2in}

In addition, the PM is an efficient source of fast electrons bunches. In the work presented in this PhD, $\sim$pC effective charge per bunch where measured, compared to $\sim$fC~\cite{Beaurepaire2015} in the same experimental conditions, but shooting in a Nitrogen gas jet. In future experiments, we will investigate in more details the fine conditions required for efficient trapping of electrons inside the reflecting laser, and therefore the tunability of the electron angular emission profile and final energy spectra with the driving laser. Knowing to what extent we can control the properties of these electrons is crucial to know if they can be used as an actual source to perform time-resolved ($\sim$fs resolution) application experiments.\\

\noindent Indeed, electrons escaping the target have a duration comparable to the driving laser pulse duration of the order of a few fs. Ultrafast Electron Diffraction using plasma-generated electrons~\cite{sciaini2011femtosecond} requires electrons between 1 and 10 MeV, spread over a few $\mu m$. After propagation, the electron bunch duration increases because of the initial energy spread. However, each monoenergetic region of the pulse conserves its initial temporal extent~\cite{faure2015concept}, which makes it suitable for the study of ultrafast crystalline dynamics through electron diffraction~\cite{tokita2010single}.