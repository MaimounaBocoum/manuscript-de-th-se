\chapter*{Résumé}

Les rayonnements électromagnétiques ou de particules énergétiques sont d'excellents outils pour sonder la matière qui nous environne. On pense par exemple à l'utilisation des rayons X pour la radiographie ou encore la diffraction pour la compréhension de la structure atomique des cristaux. Toutefois, lorsqu'il s'agit d'étudier la matière à des échelles de temps atomiques (typiquement sub-femtoseconde, $1\,\mathrm{fs}=10^{-15}\mathrm{s}$), les seules sources de rayonnement disponibles restent des installations de grande envergure comme des synchrotrons, avec une résolution temporelle de l'ordre de la picoseconde, ou les lasers à électrons libres, sources parmi les plus intenses, avec des résolutions de l'ordre de la femtoseconde. Depuis une cinquantaines d'années, de nombreux travaux ont été menés pour proposer un jour une alternative reposant sur l'interaction laser-plasma. En effet, lorsqu'un plasma interagit avec une impulsion optique intense ($> 10^{16-17}\mathrm{W/cm^2}$), il en résulte l'accélération de particules (ions,électrons) et l'émission de rayonnement électromagnétique énergétique (X-UV) cohérent. Dans le cadre de cette thèse, ont étudiera la réponse d'un plasma de densité solide lorsqu'il est stimulé par une impulsion d'une dizaine de femtosecondes.\\
 Un plasma est créé sur une cible solide avec un première impulsion de quelques $10^{14}\,\mathrm{W/cm^2}$, dite impulsion pompe. Ce plasma possède une interface avec le vide de qualité optique, si bien qu'il réfléchit en grande partie une seconde impulsion laser, dite impulsion sonde, d'intensité bien plus élevée ($\sim 10^{17-18}\,\mathrm{W/cm^2}$): on parle alors de miroir plasma. C'est la partie non réfléchie de la sonde qui interagira non-linéairement avec ce plasma pour émettre du rayonnement X-UV ou bien libérer les particules accélérées comme des électrons.  Le plasma créé avec la pompe se détend dans le vide à la vitesse du son, si bien qu'il est important de contrôler le délai pompe-sonde de manière précise. En effet, le mécanisme d'interaction avec un plasma peu détendu ($\sim\lambda/100$, donc délai pompe-sonde nul) facilitera l'émission de rayonnement X-UV, alors qu'une fois détendu d'une fraction de la longueur d'onde ($\sim\lambda/10$, donc augmentation du délai pompe-sonde), c'est l'accélération d'électrons qui prédominera. \\

\noindent Dans le premier Chapitre, nous revenons sur le mécanisme de ``Brunel'' décrivant l'interaction d'une impulsion laser avec un miroir plasma. Nous pourrons ainsi expliquer comment le plasma peut émettre une impulsion attoseconde dans la gamme X-UV à chaque cycle optique du laser incident. Ces impulsions optiques interfèrent spectralement pour donner les harmoniques de la fréquence centrale du laser incident: c'est ce qu'on appelle ``génération d'harmoniques''. Les résultats sur la génération contrôlée d'harmoniques seront présentés au chapitre 4. En particulier, on montrera comment l'introduction de couplages spatio-temporels dans le champs laser incident permet de faire ressortir des informations sur le profil temporel du train d'impulsion attosecondes issu de l'interaction. Nous effectuerons un parallèle avec des mesure type FROG (\textit{Frequency-Resolved Optical Gating}), technique de mesure bien connue pour les impulsions laser ultracourtes, fondé sur un algorithme de reconstruction. \\

\noindent Au chapitre 6, nous exposons un des résultats majeur de cette thèse: en changeant la longueur caractéristique du plasma avec lequel notre laser interagit, il est possible de façon sélective ou bien de générer des harmoniques, ou bien d'accélérer des électrons. Nous expliquons, en s'appuyant sur des simulations numériques comment la charge d'espace en surface du plasma permet d'expliquer cette observation. Cela nous conduira à nous focaliser sur le mécanisme d'accélération d'électrons au chapitre 7, notamment à travers l'observation de la distribution angulaire d'emission des électrons, ainsi que leur spectre en énergie. \\

\noindent La longueur caractéristique plasma étant de quelques fractions de longueur d'onde, aussi nous présentons au chapitre 5 une technique de mesure de l'expansion du plasma développé pendant cette thèse, basé sur l'interférométrie spatiale, et qui nous à permis de calibrer la vitesse d'expansion de notre plasma. Le principe est le suivant: un masque troué périodiquement est inséré dans le trajet de l'impulsion principale, si bien que plusieurs tâches de diffraction s'observent dans le plan focale de la parabole. Uniquement l'ordre 0 de diffraction est réfléchit par le miroir plasma, les autres étant réfléchis sur la surface du solide, si bien qu'au cours de la détente, les modulation du profil d'intensité du champs laser réfléchi, directement lié à un déphasage de l'ordre 0, permettent de remonter à l'expansion du plasma.\\

\noindent Nous décrirons au Chapitre 2, le système laser utilisé pour réaliser ces expériences sur cible solide, délivrant des impulsions de quelques millijoules, $<30\,\mathrm{fs}$ au kHz. La compréhension et le développement de cette chaine à représenté un grande partie du travail de thèse, comme par exemple l'identification d'un processus de conversion de post-pulse en pré-pulse par non linéarité. Toutefois, un seul chapitre sera consacré au laser, l'exposé de cette thèse visant à décrire de la manière la plus exhaustive possible la physique de l'interaction laser-miroir plasma.